\documentclass[times, utf8, zavrsni]{fer}
\usepackage{booktabs}

\begin{document}

\thesisnumber{000}

\title{Praćenje aktivnosti korisnika usluge računalne provjere pravopisa}

\author{Stanko Krtalić Rusendić}

\maketitle

% Ispis stranice s napomenom o umetanju izvornika rada.
% Uklonite naredbu \izvornik ako želite izbaciti tu stranicu.
% \izvornik

% Dodavanje zahvale ili prazne stranice. Ako ne želite dodati zahvalu,
% naredbu ostavite radi prazne stranice.
\zahvala{}

\tableofcontents

\chapter{Uvod}
Internetske aplikacije dnevno posjećuju na tisuće korisnika koji koriste
usluge aplikacije na različite načine. Obrasci ponašanja korisnika su bitni.
Pomoću tih podataka se može lakše upravljati i preciznije unaprijediti pojedine
usluge aplikacije.

Glavni problem je identifikacija korisnika. Iako je moguće dozvoliti korisniku
da stvori račun u Internetskoj aplikaciji, te ga tako identificirati, većina
korisnika se ne odluči stvoriti račun niti se prijavi u postojeći račun ako im
tim postupkom aplikacija ne nudi značajno proširenu funkcionalnost. Posljedica
takvog ponašanja korisnika je gubitak podataka o njihovim uzorcima korištenja
aplikacije.

Stoga je cilj ovog završnog rada analiza izvedivosti i arhitekture, te
implementacija sustav koji će omogućiti praćenje obrazaca ponašanja korisnika
sustava za računalnu provjeru pravopisa koji nisu prijavljeni ili nemaju račun u
Internetskoj aplikaciji.

U sklopu ovog završnog rada će se analizirati i objasniti dostupne metode
praćenja korisnika. Nakon toga sljedi opis i analiza arhitekture sustava.
Potom sljedi detaljan opis programskog riješenja i tehnologija korištenih u
izradi riješenja, te upute za instalaciju i pokretanje. Na posljetku će biti
opisane ideje za daljnje unapređenje programskog riješenja.

\chapter{Metode za pračenje korisnika}


\chapter{Analiza i opis arhitekture idejnog riješenja}

\chapter{Opis programskog riješenja}

\chapter{Instalacija i pokretanje}

\chapter{Ideje za daljnji razvoj projekta}

\chapter{Zaključak}
Zaključak.

\bibliography{literatura}
\bibliographystyle{fer}

\begin{sazetak}
Hrvatski akademski spelling checker (Hascheck) usluga je računalne provjere
pravopisa dostupna korisnicima putem preglednika weba. Kako bi upravljali
uslugom te uvodili poboljšanja i novosti, administratorima je važno znati
kako korisnici koriste uslugu i kako se na sjedištu ponašaju. Vaš je zadatak
opisati arhitekturu i razviti sustav koji će administratorima olakšati
upravljanje korisnicima, koristeći mehanizme koje osiguravaju protokol HTTP i
odgovarajuće funkcionalnosti preglednika.

% TODO: Add keywords
\kljucnerijeci{Ključne riječi, odvojene zarezima.}
\end{sazetak}

\engtitle{Tracking the Activity of Online Spellchecker Users}
\begin{abstract}
Abstract.

\keywords{Keywords.}
\end{abstract}

\end{document}
