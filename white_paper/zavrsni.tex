\documentclass[times, utf8, zavrsni]{fer}
\usepackage{booktabs}

\begin{document}

\thesisnumber{000}

\title{Praćenje aktivnosti korisnika usluge računalne provjere pravopisa}

\author{Stanko Krtalić Rusendić}

\maketitle

% Ispis stranice s napomenom o umetanju izvornika rada.
% Uklonite naredbu \izvornik ako želite izbaciti tu stranicu.
% \izvornik

% Dodavanje zahvale ili prazne stranice. Ako ne želite dodati zahvalu,
% naredbu ostavite radi prazne stranice.
\zahvala{}

\tableofcontents

\chapter{Uvod}
Internetske aplikacije dnevno posjećuju na tisuće korisnika koji koriste
usluge aplikacije na različite načine. Obrasci ponašanja korisnika su bitni.
Pomoću tih podataka se može lakše upravljati i preciznije unaprijediti pojedine
usluge aplikacije.

Glavni problem je identifikacija korisnika. Iako je moguće dozvoliti korisniku
da stvori račun u Internetskoj aplikaciji, te ga tako identificirati, većina
korisnika se ne odluči stvoriti račun niti se prijavi u postojeći račun ako im
tim postupkom aplikacija ne nudi značajno proširenu funkcionalnost. Posljedica
takvog ponašanja korisnika je gubitak podataka o njihovim uzorcima korištenja
aplikacije.

Stoga je cilj ovog završnog rada analiza izvedivosti i arhitekture, te
implementacija sustav koji će omogućiti praćenje obrazaca ponašanja korisnika
sustava za računalnu provjeru pravopisa koji nisu prijavljeni ili nemaju račun u
Internetskoj aplikaciji.

U sklopu ovog završnog rada će se analizirati i objasniti dostupne metode
praćenja korisnika. Nakon toga sljedi opis i analiza arhitekture sustava.
Potom sljedi detaljan opis programskog riješenja i tehnologija korištenih u
izradi riješenja, te upute za instalaciju i pokretanje. Na posljetku će biti
opisane ideje za daljnje unapređenje programskog riješenja.

\chapter{Metode za pračenje korisnika}
Kako bi pratili, tj. identificirali, korisnika potrebno je prikupiti podatke
pomoću kojih možemo jednoznačno, ili s visokom sigurnošću, ustanoviti o kojem
se korisniku radi. Navedene metode pojedinačno mogu biti dosta neprecizne,
međutim sve zajedno nam daju visoku statističku sigurnost da ćemo točno
identificirati korisnika.

\section{Metode pračenja pomoću HTTP protokola}
Internetske aplikacije koriste HTTP protokol kako bi primile zahtjev od
korisnika i kako bi mu prenijele generirani odgovor na zahtjev. Mehanizme koje
protokol implementira u svrhu optimizaciju upita, te mehanizme za perzistenciju
stanja između klijenta i poslužitelja je moguće iskoristiti u svrhu
identifikacije korisnika.

\subsection{Kolačići (eng. cookies)}
Kolačići su mehanizam perzistencije stanja.
Kolačići su tekstualne datoteke koje klijent i poslužitelje razmjenjuju pri
svakom upitu. U njih se zapisuju podaci koje klijent ili poslužitelj
žele perzistirati u slučaju greške ili za potrebe domenske logike.

Ovaj mehanizam se može iskoristiti u svrhu stvaranja 'kolačića za praćenje'.

Kolačić za praćenje (eng. tracking cookie) služi kao spremište za podatak
prema kojem možemo identificirati korisnika. (npr. identifikacijski broj ili
slijed znamenki)

Problem s kolačićem za praćenje je njegova transparentnost (korisnik
lako može otkriti njegovo postojanje), dostupnost alata za uklanjanje
kolačića, te direktiva 2009/136/EC Europske unije koja naliježe da
aplikacije moraju jasno i transparentno obavijestiti korisnika koristi li
kolačiće i u koje svrhe ih koriste.

\subsection{ETag i Last-Modified zaglavlja}
ETag i Last-Modified zaglavlja su mehanizmi optimizacije upita.
ETag i Last-Modified zaglavlja služe za dojavljivanje promjene sadržaja.
Prvotno ih postavlja poslužitelj kada šalje klijentu odgovor, a klijent ih
šalje poslužitelju pri svakom sljedećem upitu na istu putanju. Poslužitelj
može iz poslanih zaglavlja ustanoviti je li se sadržaj koji se servira
promijenio od trenutka posljednjeg klijentovog upita i ovisno o tome
odgovoriti s novim sadržajem ili s praznim odgovorom. Ovaj mehanizam
značajno ubrzava rad poslužitelja.

Prema standardu ETag može biti proizvoljan slijed znakova, dok Last-Modified
treba predstavljati datum. Međutim format datuma za Last-Modified nije
standardiziran, te je efektivno isto proizvoljan slijed od 32 znaka.

Ovi mehanizmi se ponašaju slično kao kolačići, te se mogu iskoristiti u
svrhu pohranjivanja identifikatora. Korisniku nije dostupan alat s kojim bi
mogao lagano otkloniti ova zaglavlja, niti može lako otkriti u koju svrhu se
koriste zaglavlja. Stoga je ova metoda identifikacije efikasnija od
kolačića.

\subsection{Certifikati vezani uz izvorište (eng. Origin Bound Certificates)}
poznatiji kao identifikatori kanala (eng. channelID), služe
za perzistenciju podataka između klijenta i poslužitelja na strani
poslužitelja.

Ponašanjem su identični kolačićima. Predstavljaju idealni način za
pohranjivanje identifikatora jer korisnik ne posjeduje mehanizam s kojim bi
izbrisao podatke.

\subsection{Globalna IP adresa klijenta}
Može se koristiti kao identifikator, međutim jako je nepouzdana. Naj
rasprostranjeniji oblik IP adresa su IPv4 adrese koje se koriste od sredine
1983. godine. Problem sa IPv4 standardom je maksimalni mogući broj adresa. Kako
postoji više uređaja koji pristupaju internetu nego slobodnih IPv4 adresa,
potrebno je reciklirati adrese. Tj. kada neki uređaj ne koristi njemu
dodijeljenu IP adresu ona se dodjeljuje drugom uređaju.

Stoga je ovakav mehanizam za identifikaciju korisnika izrazito nepouzdan.
Implementacijom IPv6 standarda bi bilo moguće svakom korisniku dodijeliti
unikatnu IP adresu. Međutim postojeća mrežna infrastruktura još nije spremna
za potpuni prelazak na novi standard.

\subsection{Lokalna IP adresa klijenta}
Kao jedno od rješenja problema
nedostatka IPv4 adresa se koriste prevoditelji mrežnih adresa (eng.
Network Address Translator) koji omogućuju da više korisnika iz iste mreže
pristupa Internetu pod istom IP adresom.

Lokalne mreže nisu podložne čestim promjenama. Stoga se korisnikova IP
adresa unutar lokalne mreže rijetko mijenja, te se ta informacija može
koristiti u svrhu identifikacije korisnika. Međutim puno korisnika ima
sličnu konfiguraciju mreže te je teško jednoznačno identificirati nekog
korisnika.

\section{Metode pračenja pomoću Internetskog preglednika}
Korisnici koriste Internetske preglednike (eng. 'browser') kako bi ostvarili
interakciju s aplikacijom, te kako bi na jednostavan i vizualno razumljiv način
konzumirali HTML datoteke koje im ona šalje. Moderni preglednici implementiraju
razne tehnologije i mehanizme pomoću kojih omogućuju Internetskoj aplikaciji da
servira interaktivni sadržaj i optimira zahtjeve za vanjskim resursima. Slično
kao i kod mehanizama HTTP protokola, ove mehanizme je moguće iskoristiti za
pohranjivanje identifikatora.

\subsection{Lokalno pohranjeni podaci (eng. Local Storage)}. Moderni
preglednici dozvoljavaju korisniku da pomoću JavaScript, Flash i Silverlight
skripti korisniku ponudi interaktivni sadržaj koji se perzistira.

Slično kao i kod kolačića u HTTP protokolu, moguće je pohraniti podatke
na korisnikovo računalo te u njih pohraniti identifikator. Iako je ovakav
način pohrane podataka manje transparentan od kolačića, korisnicima su lako
dostupni alati za brisanje lokalno pohranjenih podataka.

\subsection{Podaci pohranjeni u svrhu brzog pristupa (eng. Cached objects)}.
Moderni preglednici pohranjuju podatke poput slika i CSS datoteka lokalno
kako bi korisniku mogli brže prikazati Internetsku stranicu. Pomoću
stenografskih metoda je moguće stvoriti sliku koja u sebi sadrži
identifikator i predati ju pregledniku da ju pohrani.

\subsection{Pomak takta sistemskog sata}
Svako računalo ima oscilacijski kristal koji generira takt jezgre procesora
i sistemskog sata. Svi kristali imaju pomak od njihove rezonantne
frekvencije koji je unikatan za taj kristal.

Pomak frekvencije je moguće izmjeriti kroz preglednik i time jedinstveno
identificirati korisnikovo računalo.

\subsection{Informacije o komponentama računala}
Korisnika je moguće identificirati prema popisu komponenta njegovog
računala.

Ovakav način identifikacije korisnika je izrazito nepouzdan jer danas
postoji ogroman broj sličnih ili identičnih računalnih konfiguracija.

\chapter{Analiza i opis arhitekture idejnog riješenja}
Programsko rješenje se mora integrirati s uslugom računalne provjere
pravopisa. Kako nam je arhitektura i izvedba usluge za provjeru pravopisa
nepoznata nalaže se arhitektura servisa.

Odnosno, programsko rješenje će biti Internetska aplikacija koju će druge
aplikacije koristiti kao servis za identifikaciju korisnika.

\section{Koponente sustava}
Sustav se sastoji od tri komponente. Programske biblioteke koja se izvodi u
pregledniku korisnika (u daljnjem tekstu klijentska biblioteka),
biblioteke koja se izvodi na poslužitelju (u daljnjem tekstu poslužiteljska
biblioteka), te Internetske aplikacije.

\subsection{Internetska aplikacija}
Internetska aplikacija je središnja komponenta sustava. Klijentska i
poslužiteljska biblioteka razgovaraju s Internetskom aplikacijom preko HTTP-a
kako bi joj dojavile podatke o korisniku prema kojima će ona pronaći ili
generirati identifikator, te uputiti biblioteku koje podatke da pohrani ili
servira korisniku.

\subsection{Klijentska biblioteka}
Klijentska aplikacija prikuplja podatke o pregledniku i računalu korisnika,
zapisuje ih u format koji Internetska aplikacija može interpretirati i šalje na
obradu.

Kada dobije od Internetske aplikacije odgovor s identifikatorom ga zapisuje
u sve dostupne medije za pohranu u klijentovom pregledniku, te dojavljuje
usluzi za provjeru pravopisa identifikator korisnika i putanju kojoj je
pristupio.

\subsection{Poslužiteljska biblioteka}
Poslužiteljska biblioteka pri svakom korisnikovom upitu ekstrahira podatke iz
zaglavlja HTTP upita i šalje ih na obradu Internetskoj aplikaciji.

Internetska aplikacija odgovara biblioteci sa identifikatorom korisnika, ako
on postoji, i podacima koje treba postaviti u zaglavlje HTTP odgovora.

\section{Dijagram toka}
\section{ER dijagram}

\chapter{Opis programskog riješenja}

\chapter{Instalacija i pokretanje}

\chapter{Ideje za daljnji razvoj projekta}

\chapter{Zaključak}
Zaključak.

\bibliography{zavrsni}
\bibliographystyle{fer}

\begin{sazetak}
TODO

% TODO: Add keywords
\kljucnerijeci{Ključne riječi, odvojene zarezima.}
\end{sazetak}

\engtitle{Tracking the Activity of Online Spellchecker Users}
\begin{abstract}
Abstract.

\keywords{Keywords.}
\end{abstract}

\end{document}
