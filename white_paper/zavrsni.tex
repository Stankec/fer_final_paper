\documentclass[times, utf8, zavrsni]{fer}
\usepackage{booktabs}
\usepackage{listings}
\graphicspath{ {images/} }

\begin{document}

\thesisnumber{000}

\title{Praćenje aktivnosti korisnika usluge računalne provjere pravopisa}

\author{Stanko Krtalić Rusendić}

\maketitle

% Ispis stranice s napomenom o umetanju izvornika rada.
% Uklonite naredbu \izvornik ako želite izbaciti tu stranicu.
% \izvornik

% Dodavanje zahvale ili prazne stranice. Ako ne želite dodati zahvalu,
% naredbu ostavite radi prazne stranice.
\zahvala{}

\tableofcontents

\chapter{Uvod}
Internetske aplikacije dnevno posjećuju na tisuće korisnika koji koriste
usluge aplikacije na različite načine. Obrasci ponašanja korisnika su bitni.
Pomoću tih podataka se može lakše upravljati i preciznije unaprijediti pojedine
usluge aplikacije.

Glavni problem je identifikacija korisnika. Iako je moguće dozvoliti korisniku
da stvori račun u Internetskoj aplikaciji, te ga tako identificirati, većina
korisnika se ne odluči stvoriti račun niti se prijavi u postojeći račun ako im
tim postupkom aplikacija ne nudi značajno proširenu funkcionalnost. Posljedica
takvog ponašanja korisnika je gubitak podataka o njihovim uzorcima korištenja
aplikacije.

Stoga je cilj ovog završnog rada analiza izvedivosti i arhitekture, te
implementacija sustav koji će omogućiti praćenje obrazaca ponašanja korisnika
sustava za računalnu provjeru pravopisa koji nisu prijavljeni ili nemaju račun u
Internetskoj aplikaciji.

U sklopu ovog završnog rada će se analizirati i objasniti dostupne metode
praćenja korisnika. Nakon toga slijedi opis i analiza arhitekture sustava.
Potom slijedi detaljan opis programskog rješenja i tehnologija korištenih u
izradi rješenja, te upute za instalaciju i pokretanje. Na posljetku će biti
opisane ideje za daljnje unapređenje programskog rješenja.

\chapter{Metode za praćenje korisnika}
Kako bi pratili, tj. identificirali, korisnika potrebno je prikupiti podatke
pomoću kojih možemo jednoznačno, ili s visokom sigurnošću, ustanoviti o kojem
se korisniku radi. Navedene metode pojedinačno mogu biti dosta neprecizne,
međutim sve zajedno nam daju visoku statističku sigurnost da ćemo točno
identificirati korisnika.

\section{Metode praćenja pomoću HTTP protokola}
Internetske aplikacije koriste HTTP protokol kako bi primile zahtjev od
korisnika i kako bi mu prenijele generirani odgovor na zahtjev. Mehanizme koje
protokol implementira u svrhu optimizaciju upita, te mehanizme za perzistenciju
stanja između klijenta i poslužitelja je moguće iskoristiti u svrhu
identifikacije korisnika.

\subsection{Kolačići (eng. cookies)}
Kolačići su mehanizam perzistencije stanja.
Kolačići su tekstualne datoteke koje klijent i poslužitelje razmjenjuju pri
svakom upitu. U njih se zapisuju podaci koje klijent ili poslužitelj
žele perzistirati u slučaju greške ili za potrebe domenske logike.

Ovaj mehanizam se može iskoristiti u svrhu stvaranja 'kolačića za praćenje'.

Kolačić za praćenje (eng. tracking cookie) služi kao spremište za podatak
prema kojem možemo identificirati korisnika. (npr. identifikacijski broj ili
slijed znamenki)

Problem s kolačićem za praćenje je njegova transparentnost (korisnik
lako može otkriti njegovo postojanje), dostupnost alata za uklanjanje
kolačića, te direktiva 2009/136/EC Europske unije koja naliježe da
aplikacije moraju jasno i transparentno obavijestiti korisnika koristi li
kolačiće i u koje svrhe ih koriste.

\subsection{ETag i Last-Modified zaglavlja}
ETag i Last-Modified zaglavlja su mehanizmi optimizacije upita.
ETag i Last-Modified zaglavlja služe za dojavljivanje promjene sadržaja.
Prvotno ih postavlja poslužitelj kada šalje klijentu odgovor, a klijent ih
šalje poslužitelju pri svakom sljedećem upitu na istu putanju. Poslužitelj
može iz poslanih zaglavlja ustanoviti je li se sadržaj koji se servira
promijenio od trenutka posljednjeg klijentovog upita i ovisno o tome
odgovoriti s novim sadržajem ili s praznim odgovorom. Ovaj mehanizam
značajno ubrzava rad poslužitelja.

Prema standardu ETag može biti proizvoljan slijed znakova, dok Last-Modified
treba predstavljati datum. Međutim format datuma za Last-Modified nije
standardiziran, te je efektivno isto proizvoljan slijed od 32 znaka.

Ovi mehanizmi se ponašaju slično kao kolačići, te se mogu iskoristiti u
svrhu pohranjivanja identifikatora. Korisniku nije dostupan alat s kojim bi
mogao lagano otkloniti ova zaglavlja, niti može lako otkriti u koju svrhu se
koriste zaglavlja. Stoga je ova metoda identifikacije efikasnija od
kolačića.

\subsection{Certifikati vezani uz izvorište (eng. Origin Bound Certificates)}
poznatiji kao identifikatori kanala (eng. channelID), služe
za perzistenciju podataka između klijenta i poslužitelja na strani
poslužitelja.

Ponašanjem su identični kolačićima. Predstavljaju idealni način za
pohranjivanje identifikatora jer korisnik ne posjeduje mehanizam s kojim bi
izbrisao podatke.

\subsection{Globalna IP adresa klijenta}
Može se koristiti kao identifikator, međutim jako je nepouzdana. Naj
rasprostranjeniji oblik IP adresa su IPv4 adrese koje se koriste od sredine
1983. godine. Problem s IPv4 standardom je maksimalni mogući broj adresa. Kako
postoji više uređaja koji pristupaju internetu nego slobodnih IPv4 adresa,
potrebno je reciklirati adrese. Tj. kada neki uređaj ne koristi njemu
dodijeljenu IP adresu ona se dodjeljuje drugom uređaju.

Stoga je ovakav mehanizam za identifikaciju korisnika izrazito nepouzdan.
Implementacijom IPv6 standarda bi bilo moguće svakom korisniku dodijeliti
unikatnu IP adresu. Međutim postojeća mrežna infrastruktura još nije spremna
za potpuni prelazak na novi standard.

\subsection{Lokalna IP adresa klijenta}
Kao jedno od rješenja problema
nedostatka IPv4 adresa se koriste prevoditelji mrežnih adresa (eng.
Network Address Translator) koji omogućuju da više korisnika iz iste mreže
pristupa Internetu pod istom IP adresom.

Lokalne mreže nisu podložne čestim promjenama. Stoga se korisnikova IP
adresa unutar lokalne mreže rijetko mijenja, te se ta informacija može
koristiti u svrhu identifikacije korisnika. Međutim puno korisnika ima
sličnu konfiguraciju mreže te je teško jednoznačno identificirati nekog
korisnika.

\section{Metode praćenja pomoću Internetskog preglednika}
Korisnici koriste Internetske preglednike (eng. 'browser') kako bi ostvarili
interakciju s aplikacijom, te kako bi na jednostavan i vizualno razumljiv način
konzumirali HTML datoteke koje im ona šalje. Moderni preglednici implementiraju
razne tehnologije i mehanizme pomoću kojih omogućuju Internetskoj aplikaciji da
servira interaktivni sadržaj i optimira zahtjeve za vanjskim resursima. Slično
kao i kod mehanizama HTTP protokola, ove mehanizme je moguće iskoristiti za
pohranjivanje identifikatora.

\subsection{Lokalno pohranjeni podaci (eng. Local Storage)}. Moderni
preglednici dozvoljavaju korisniku da pomoću JavaScript, Flash i Silverlight
skripti korisniku ponudi interaktivni sadržaj koji se perzistira.

Slično kao i kod kolačića u HTTP protokolu, moguće je pohraniti podatke
na korisnikovo računalo te u njih pohraniti identifikator. Iako je ovakav
način pohrane podataka manje transparentan od kolačića, korisnicima su lako
dostupni alati za brisanje lokalno pohranjenih podataka.

\subsection{Podaci pohranjeni u svrhu brzog pristupa (eng. Cached objects)}.
Moderni preglednici pohranjuju podatke poput slika i CSS datoteka lokalno
kako bi korisniku mogli brže prikazati Internetsku stranicu. Pomoću
stenografskih metoda je moguće stvoriti sliku koja u sebi sadrži
identifikator i predati ju pregledniku da ju pohrani.

\subsection{Pomak takta sistemskog sata}
Svako računalo ima oscilacijski kristal koji generira takt jezgre procesora
i sistemskog sata. Svi kristali imaju pomak od njihove rezonantne
frekvencije koji je unikatan za taj kristal.

Pomak frekvencije je moguće izmjeriti kroz preglednik i time jedinstveno
identificirati korisnikovo računalo.

\subsection{Informacije o komponentama računala}
Korisnika je moguće identificirati prema popisu komponenta njegovog
računala.

Ovakav način identifikacije korisnika je izrazito nepouzdan jer danas
postoji ogroman broj sličnih ili identičnih računalnih konfiguracija.

\chapter{Analiza i opis arhitekture idejnog rješenja}
Programsko rješenje se mora integrirati s uslugom računalne provjere
pravopisa. Kako nam je arhitektura i izvedba usluge za provjeru pravopisa
nepoznata nalaže se arhitektura servisa.

Odnosno, programsko rješenje će biti Internetska aplikacija koju će druge
aplikacije koristiti kao servis za identifikaciju korisnika.

\section{Komponente sustava}
Sustav se sastoji od tri komponente. Programske biblioteke koja se izvodi u
pregledniku korisnika (u daljnjem tekstu klijentska biblioteka),
biblioteke koja se izvodi na poslužitelju (u daljnjem tekstu poslužiteljska
biblioteka), te Internetske aplikacije.

\subsection{Internetska aplikacija}
Internetska aplikacija je središnja komponenta sustava. Klijentska i
poslužiteljska biblioteka razgovaraju s Internetskom aplikacijom preko HTTP-a
kako bi joj dojavile podatke o korisniku prema kojima će ona pronaći ili
generirati identifikator, te uputiti biblioteku koje podatke da pohrani ili
servira korisniku.

Aplikacija treba implementirati programsko sučelje preko kojeg može razgovarati
s bibliotekama.

Ustroj baze podataka je izrazito bitan. Kako se radi o potencijalno velikom setu
podataka koje će aplikacija morati pretraživati bi svi primarni ključevi trebali
biti tipa UUID. Te je izrazito bitno dobro postavljanje indeksa na kolumne
tablica. Optimalna implementacija baze podataka bi bila PostgreSQL baza zbog
brzine i mogućnosti pohranjivanja i pretraživanja indeksiranih no-SQL podataka
što omogućuje pohranjivanje kompleksnih seta podataka.

Aplikacija za svakog korisnika stvara profil. Korisnikov profil se asocira uz
njegov 'otisak prsta' koji predstavljaju podaci o CPU, GPU i kristalu računala
koje posjećuje servis. Profil se također asocira uz oznake (eng. tags) koji
predstavljaju oznake HTTP zaglavlja i putanju na koju je to zaglavlje bilo
poslano. Konačno, profil se asocira i uz ip adresu s koje je upit napravljen.

\subsection{Klijentska biblioteka}
Klijentska aplikacija prikuplja podatke o pregledniku i računalu korisnika,
zapisuje ih u format koji Internetska aplikacija može interpretirati i šalje na
obradu.

Kada dobije od Internetske aplikacije odgovor koji sadrži identifikator,
zapiše ga u sve dostupne medije za pohranu u klijentovom pregledniku, te
dojavljuje usluzi za provjeru pravopisa identifikator korisnika i putanju kojoj
je pristupio.

Kako svi moderni preglednici implementiraju i dozvoljavaju izvršavanje
JavaScript skripti na korisnikovom uređaju, implementacija ove biblioteke će
biti u programskom jeziku JavaScript.

Problem izvršavanja logike na klijentovom pregledniku je sposobnost korisnika
da ugasi mogućnost izvršavanja JavaScript skripti u pregledniku. Ovakva
situacija nije česta, korisnik tim činom gubi cjelovitu funkcionalnost
usluge za provjeru pravopisa. Međutim domišljat korisnik bi mogao iz izvornog
JavaScript koda servisa iščitati putanje i tipove upita koje mora napraviti kako
bi servisu predao tekst na ispravljanje. Kao mehanizam detekcije takvog načina
korištenja servisa postoji poslužiteljska biblioteka.

\subsection{Poslužiteljska biblioteka}
Poslužiteljska biblioteka pri svakom korisnikovom upitu ekstrahira podatke iz
zaglavlja HTTP upita i šalje ih na obradu Internetskoj aplikaciji.

Internetska aplikacija odgovara biblioteci s identifikatorom korisnika, ako
on postoji, i podacima koje treba postaviti u zaglavlje HTTP odgovora.

Poslužiteljska biblioteka može biti implementirana u bilo kojem programskom
jeziku.

Poslužiteljska biblioteka služi kao mehanizam za detekciju korisnika koji nisu
pokrenuli klijentsku biblioteku, te korisnika koji pokušavaju zaobići
transparentne mehanizme identifikacije.

\section{Dijagram toka}
\includegraphics[width=\textwidth]{flow_chart}
Dijagram toka prikazuje tok izvođenja programa od strane klijentskog
preglednika, servisa za ispravljanje pravopisa, Internetske aplikacije,
klijentske i poslužiteljske biblioteke.

\section{ER dijagram}
\includegraphics[width=\textwidth]{er_diagram}
ER dijagram prikazuje ustroj baze podataka.

\chapter{Opis programskog rje��enja}
Kao što je i opisano u prethodnom poglavlju. Programsko rješenje se sastoji od
tri dijela.

\section{Internetska aplikacija}
Internetska aplikacija implementirana je u programskom jeziku Ruby pomoću Ruby
on Rails aplikacijskog okvira (eng. framework). Ruby on Rails (u daljnjem tekstu
Rails) nam nudi slojeve modela, pogleda i kontrolera (eng. MVC - Model View
Controller), te mnoge biblioteke koje nas štite od sigurnosnih propusta poput
injekcije SQL naredbi (eng. SQL injection), među straničnog skriptiranja (eng.
Cross Site Scripting), preljeva kolačića (eng. Cookie overflow) i mnogih drugih.
Osim što nam nudi visoki sigurnosni standard nam Rails nudi obradu upita kroz
posredničke programe (eng. middleware) i bogati ekosustav gotovih biblioteka.

U Ruby on Rails aplikacijama model reprezentira podatak iz baze podataka.
Manipulacijom atributa modela se direktno manipuliraju podaci u bazi podataka.
Kroz modele možemo i raditi upite nad bazom i dobivati kolekcije modela koji
zadovoljavaju dani SQL upit. Rails se automatski brine o optimizaciji upita.

Kontroleri u Ruby on Rails aplikacijama služe kao posrednički sloj između
pogleda i modela. Rails aplikacija ne mora nužno odgovoriti s pogledom. Iako
kontroleri služe kao posrednički sloj oni su u potpunosti zaslužni za formiranje
odgovora korisniku.

Pogledi su HTML datoteke koje se poslužuje kao odgovor na upit. Rails koristi
ugrađeni Ruby kod (eng. Embeded Ruby - ERB) kako bi dinamički izmijenio HTML
koji poslužuje klijentima.

Rails aplikacija ima četiri modela. Profil u koji se pohranjuje identifikator
korisnika. Zatim 'otiske prstiju' u koji se pohranjuju podaci specifični za
korisnikovo računalo. IP adrese u koje se pohranjuju IP adrese s kojima je
korisnik napravio zahtjeve na servis. I konačno, oznake za praćenje (eng.
tracking tags) gdje se pohranjuju oznake HTTP zaglavlja.

Postoji samo jedan kontroler. Taj kontroler je zaslužan za komunikaciju između
aplikacije i biblioteka. Kako kontroler prima i odgovara na zahtjeve s JSON
serijaliziranim objektima nema potrebe za pogledima.

Logika za identifikaciju korisnika pomoću predanih podataka se nalazi u
servisnim objektima (eng. service objects) koji razgovaraju s objektima za
izgradnju SQL upita (eng. query objects). Ovakav način korištenja je proširenje
obične MVC arhitekture. Cilj ovakvog proširenja je apstrakcija implementacije
poslovne logike i baze podataka. Odnosno, cilj je omogućiti nam da možemo
proizvoljno centralizirano mijenjati logiku, bez da moramo mijenjati kod u više
datoteka.

\section{Poslužiteljska biblioteka}
Poslužiteljska biblioteka je implementirana kao Ruby datoteka, odnosno Ruby on
Rails servisni objekt. Kontroler servisu predaje zaglavlja koja je dobio u
zahtjevu, a servis mu odgovara sa zaglavljima koje mora postaviti u odgovoru,
te identifikatorom korisnika ako je korisnik uspješno identificiran.

\section{Klijentska biblioteka}
Klijentska biblioteka implementirana je kao JavaScript objekt. Nakon što se
doda u HTML koji se servira korisniku, potrebno je konfigurirati putanju
Rails aplikacije i putanju na koju se dojavljuje identifikator korisnika.

Biblioteka koristi evercookie biblioteku kako bi perzistirala podatke u
klijentovom pregledniku.

\chapter{Instalacija i pokretanje}
Za pokretanje programskog rješenja potrebno je računalo s operativnim sustavom
koji implementira POSIX ili MS-DOS standard na kojem je instaliran Ruby 2.3.0
virtualni stroj, te pristup PosgreSQL bazi podataka.

\section{UNIX}
\subsection{Ruby}
Dok je aplikaciju moguće upogoniti pomoću sistemske instalacije Ruby virtualnog
stroja to nije preporučljivo za produkcijska okruženja jer ne ostavlja mogućnost
pokretanja više aplikacija koje koriste različite verzije virtualnog stroja.

Preporučena metoda instalacije virtualnog stroja je pomoću menedžera verzija.
Rbenv je jedan od najpopularnijih manedžera verzija za Ruby virtualni stroj.
Kako bi ga instalirali potreban nam je Git sustav za verzioniranje.

Na debian distribucijama je potrebno izvršiti naredbu koja će na sustav
instalirati Git:

\begin{lstlisting}

apt-get update
sudo apt-get install build-essential
apt-get install git

\end{lstlisting}

Potom je potrebno preuzeti rbenv pomoću Gita i instalirati ruby-build plugin:

\begin{lstlisting}

git clone https://github.com/rbenv/rbenv.git ~/.rbenv
cd ~/.rbenv && src/configure && make -C src

# za sustave koji koriste ZSH terminal
echo 'export PATH="$HOME/.rbenv/bin:$PATH"' >> ~/.zshrc

# za Ubuntu Desktop operativne sustave
echo 'export PATH="$HOME/.rbenv/bin:$PATH"' >> ~/.bashrc

# za ostale operativne sustave
echo 'export PATH="$HOME/.rbenv/bin:$PATH"' >> ~/.bash_profile

~/.rbenv/bin/rbenv init
git clone https://github.com/rbenv/ruby-build.git \
~/.rbenv/plugins/ruby-build
rbenv install 2.3.0
rbenv global 2.3.0

\end{lstlisting}

\subsection{PostgreSQL}
Za instalaciju PosgreSQL baze podataka potrebno je izvršiti sljedeće naredbe:

\begin{lstlisting}

sudo sh -c 'echo "deb http://apt.postgresql.org/pub/repos/apt/\
`lsb_release -cs`-pgdg main" >> /etc/apt/sources.list.d/pgdg.list'
wget -q https://www.postgresql.org/media/keys/ACCC4CF8.asc -O - |\
sudo apt-key add -
sudo apt-get update
sudo apt-get install postgresql postgresql-contrib

\end{lstlisting}

\subsection{Aplikacija}
Kako bi se instalirala aplikacija potrebno ju je samo kopirati na računalo.
To je moguće napraviti kopiranjem izvornog koda za CD-a ili pomoću Gita te
pokrenuti sljedeće naredbe iz direktorija u kojem se nalazi aplikacija:

\begin{lstlisting}

  # ako zelite kopirati aplikaciju pomocu gita
  cd
  cd Desktop
  git clone git@github.com:Stankec/fer_final_paper.git
  cd fer_final_paper

  gem install bundler
  cd web_app
  bundle install

\end{lstlisting}

Konačno možemo pokrenuti aplikaciju i demo aplikaciju koja implementira
biblioteke:

\begin{lstlisting}

  cd web_app
  rails s -p 3000

\end{lstlisting}

Ovim naredbama smo podignuli dvije web aplikacije. Aplikaciju opisanu u
programskom rješenju i demonstracijsku aplikaciju koja implementira klijentsku
i poslužiteljsku biblioteku.

Aplikacija koja predstavlja programsko rješenje nalazi se na URL adresi
http://localhost:3000

\section{MS-DOS}
\subsection{Ruby}
Instalirati Ruby na Windows operativnim sustavima je najlak��e napraviti pomoću
instalacijskog čarobnjaka kojeg se može preuzeti s ove poveznice:
http://rubyinstaller.org/

\subsection{PostgreSQL}
PostgreSQL bazu podataka je najlakše instalirati kroz instalacijski čarobnjak
dostupan na ovoj poveznici https://www.postgresql.org/download/

\subsection{Aplikacija}
Kao i kod instalacije na POSIX operativne sustave potrebno je kopirati izvorni
kod aplikacije na računalo, otvoriti Command prompt i pomaknuti se u direktorij
aplikacije i potom izvršiti sljedeće naredbe:

\begin{lstlisting}

  gem install bundler
  cd web_app
  bundle install
  rails s -p 3000

\end{lstlisting}

Ovim naredbama smo podignuli dvije web aplikacije. Aplikaciju opisanu u
programskom rješenju i demonstracijsku aplikaciju koja implementira klijentsku
i poslužiteljsku biblioteku.

Aplikacija koja predstavlja programsko rješenje nalazi se na URL adresi
http://localhost:3000

\chapter{Ideje za daljnji razvoj projekta}
Programsko rješenje se može dalje unaprjeđivati. U nastavku je nekoliko ideja
koje bi mogle značajno unaprijediti programsko rješenje:

\section{Prijava korisnika}
U sustav bi se mogao implementirati sustav za stvaranje korisničkih računa za
pojedinačne korisnike i za firme. To bi omogućilo da se sustav monetizira
iznajmljivanjem odnosno plaćanjem pristupa.

\section{Migracija podataka u sustav za obradu velike količine podataka}
Za programsko riješenje bi se moglo reći da ima naivnu implementaciju dohvata,
pohranjivanja i agregacije podataka. Makar će ovakav sustav zadovoljiti potrebe
servisa za programsko ispravljanje pravopisa ne bi ga bilo moguće skalirati na
servise koji primaju više upita. Kako bi se otklonio ovaj problem bilo trebalo
bi periodički brisati podatke iz PostgreSQL baze i ostaviti samo agregati
obrisanih podataka. Obrisane podatke bi bilo korisno migrirati u bazu podataka
za obradu velikog broja podataka (eng. Big Data Database). U takvoj
konfiguraciji bi sustav imao dvije baze podataka. PosgreSQL bazu koju bi
koristio za identifikaciju korisnika u realnom vremenu i bazu podataka za veliki
broj podataka koju bi koristio za generiranje statističkih modela. Generirani
modeli bi se mogli iskoristiti za treniranje neuralne mreže koja bi mogla prema
danim podacima identificirati korisnika.

\section{Prikupljanje više podataka}
U trenutnom programskom rješenju se ne prikupljaju svi mogući podaci od
korisnika već samo podaci potrebni za identifikaciju korisnika. Sustav bi se
mogao unaprijediti pohranjivanjem dodatnih podataka koji bi se mogli iskoristiti
u svrhu analize demografije korisnika.

\section{Praćenje između uređaja}
Praćenje između uređaja je moguće izvesti kombiniranjem tradicionalnog principa
prijave u korisnički račun s predstavljenim programskim rješenjem. U trenutku
prijave korisnika u sustav bi se Internetskoj aplikaciji dojavio njegov
identifikator u sustavu u koji se prijavio. Kako se korisnik prijavljuje na
različite uređaje bi se u Internetskoj aplikaciji njegov profil povezao s
podacima koje je sustav dobio od istog korisnika s drugih uređaja, te bi mogao
identificirati uređaj uz korisnika iako korisnik nije prijavljen u sustav.

\chapter{Zaključak}
Kao što se moglo zaključiti kroz dosadašnja poglavlja ovog rada, praćenje
aktivnosti korisnika usluge računalne provjere pravopisa je izvedivo i
predstavlja neintruzivan oblik agregacije podataka u svrhu unapređenja usluge.

Iako sustav ne garantira da će točno identificirati korisnika. On to čini sa
dovoljno velikom statističkom sigurnošću da se može koristiti u produkcijskim
okruženjima.

Uz daljnja unaprjeđenja sustava koja su opisana u prethodnom poglavlju bi
se povećala pouzdanost sustava i sposobnost sustava da generira relevantne
statističke podatke i modele koji bi se mogli iskoristiti u svrhu unapređenja
usluge računalne provjere pravopisa.

\bibliography{zavrsni}
\bibliographystyle{fer}

\begin{sazetak}
Praćenje aktivnosti korisnika usluge računalne provjere pravopisa.

Procjena izvedivosti, te opis arhitekture i izvedbe sustava za praćenje
aktivnosti korisnika na usluzi računalne provjere pravopisa poput Hrvatskog
akademskog spelling checkera (Hascheck).

\kljucnerijeci{praćenje aktivnosti korisnika, identifikacija korisnika, praćenje
korisnika, Internetska aplikacija}
\end{sazetak}

\engtitle{Tracking the Activity of Online Spellchecker Users}
\begin{abstract}
Tracking the Activity of Online Spellchecker Users.

Feasibility assessment, architecture and implementation of a system for
monitoring user activity on a online spell check like the Croatian
academic spelling checker (Hascheck).

\keywords{user activity tracking, user identification, user tracking,
  web application}
\end{abstract}

\end{document}
