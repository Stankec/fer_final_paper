\documentclass[times, utf8, zavrsni]{fer}
\usepackage{booktabs}

\begin{document}

\thesisnumber{000}

\title{Praćenje aktivnosti korisnika usluge računalne provjere pravopisa}

\author{Stanko Krtalić Rusendić}

\maketitle

% Ispis stranice s napomenom o umetanju izvornika rada.
% Uklonite naredbu \izvornik ako želite izbaciti tu stranicu.
% \izvornik

% Dodavanje zahvale ili prazne stranice. Ako ne želite dodati zahvalu,
% naredbu ostavite radi prazne stranice.
\zahvala{}

\tableofcontents

\chapter{Uvod}
Internetske aplikacije dnevno posjećuju na tisuće korisnika koji koriste
usluge aplikacije na različite načine. Obrasci ponašanja korisnika su bitni.
Pomoću tih podataka se može lakše upravljati i preciznije unaprijediti pojedine
usluge aplikacije.

Glavni problem je identifikacija korisnika. Iako je moguće dozvoliti korisniku
da stvori račun u Internetskoj aplikaciji, te ga tako identificirati, većina
korisnika se ne odluči stvoriti račun niti se prijavi u postojeći račun ako im
tim postupkom aplikacija ne nudi značajno proširenu funkcionalnost. Posljedica
takvog ponašanja korisnika je gubitak podataka o njihovim uzorcima korištenja
aplikacije.

Stoga je cilj ovog završnog rada analiza izvedivosti i arhitekture, te
implementacija sustav koji će omogućiti praćenje obrazaca ponašanja korisnika
sustava za računalnu provjeru pravopisa koji nisu prijavljeni ili nemaju račun u
Internetskoj aplikaciji.

U sklopu ovog završnog rada će se analizirati i objasniti dostupne metode
praćenja korisnika. Nakon toga sljedi opis i analiza arhitekture sustava.
Potom sljedi detaljan opis programskog riješenja i tehnologija korištenih u
izradi riješenja, te upute za instalaciju i pokretanje. Na posljetku će biti
opisane ideje za daljnje unapređenje programskog riješenja.

\chapter{Metode za pračenje korisnika}
Kako bi pratili, tj. identificirali, korisnika potrebno je prikupiti podatke
pomoću kojih možemo jednoznačno, ili s visokom sigurnošću, ustanoviti o kojem
se korisniku radi.

Internetske aplikacije koriste HTTP protokol kako bi primile zahtjev od
korisnika i kako bi mu prenijele generirani odgovor na zahtjev. Mehanizme koje
protokol implementira u svrhu optimizaciju upita, te mehanizme za perzistenciju
stanja između klijenta i poslužitelja je moguće iskoristiti u svrhu
identifikacije korisnika.

\begin{enumerate}
  \item \textbf{Kolačići} (eng. cookies) su mehanizam perzistencije stanja.
    Kolačići su tekstualne datoteke koje klijent i poslužitelje razmjenjuju pri
    svakom upitu. U njih se zapisuju podaci koje klijent ili poslužitelj
    žele perzistirati u slučaju greške ili za potrebe domenske logike.

    Ovaj mehanizam se može iskoristiti u svrhu stvaranja 'kolačića za praćenje'.

    Kolačić za praćenje (eng. tracking cookie) služi kao spremište za podatak
    prema kojem možemo identificirati korisnika. (npr. identifikacijski broj ili
    sljed znamenki)

    Problem sa kolačićem za praćenje je njegova transparentnost (korisnik
    lako može otkriti njegovo postojanje), dostupnost alata za uklanjanje
    kolačića, te direktiva 2009/136/EC Europske unije koja naliježe da
    aplikacije moraju jasno i transparentno obavjestiti korisnika koristi li
    kolačiće i u koje svrhe ih koriste.

  \item \textbf{ETag i Last-Modified zaglavlja} su mehanizmi optimizacije upita.
    ETag i Last-Modified zaglavlja služe za dojavljivanje promjene sadržaja.
    Prvotno ih postavlja poslužitelj kada šalje klijentu odgovor, a klijent ih
    šalje poslužitelju pri svakom sljedećem upitu na istu putanju. Poslužitelj
    može iz poslanih zaglavlja ustanoviti je li se sadržaj koji se servira
    promjenio od trenutka posljednjeg klijentovog upita i ovisno o tome
    odgovoriti s novim sadržajem ili s praznim odgovorom. Ovaj mehanizam
    značajno ubrzava rad poslužitelja.

    Prema standardu ETag može biti proizvoljan sljed znakova, dok Last-Modified
    treba predstavljati datum. Međutim format datuma za Last-Modified nije
    standardiziran, te je efektivno isto proizvoljan sljed od 32 znaka.

    Ovi mehanizmi se ponašaju slićno kao kolačići, te se mogu iskoristit u
    svrhu pohranjivanja identifikatora. Korisniku nije dostupan alat s kojim bi
    mogao lagano otkloniti ova zaglavlja, niti može lako otkriti u koju svrhu se
    koriste zaglavlja. Stoga je ova metoda identifikacije efikasnija od
    kolačića.

  \item \textbf{Certifikati vezani uz izvorište (eng. Origin Bound
      Certificates)} poznatiji kao identifikatori kanala (eng. channelID), služe
    za perzistenciju podataka između klijenta i poslužitelja na strani
    poslužitelja.

    Ponašanjem su identični kolačićima. Predstavljaju idealni način za
    pohranjivanje identifikatora jer korisnik ne posjeduje mehanizam s kojim bi
    izbrisao podatke.
\end{enumerate}

Korisnici koriste Internetske preglednike (eng. 'browser') kako bi ostvarili
interakciju sa aplikacijom, te kako bi na jednostavan i vizualno razumljiv način
konzumirali HTML datoteke koje im ona šalje.

\chapter{Analiza i opis arhitekture idejnog riješenja}

\chapter{Opis programskog riješenja}

\chapter{Instalacija i pokretanje}

\chapter{Ideje za daljnji razvoj projekta}

\chapter{Zaključak}
Zaključak.

\bibliography{literatura}
\bibliographystyle{fer}

\begin{sazetak}
Hrvatski akademski spelling checker (Hascheck) usluga je računalne provjere
pravopisa dostupna korisnicima putem preglednika weba. Kako bi upravljali
uslugom te uvodili poboljšanja i novosti, administratorima je važno znati
kako korisnici koriste uslugu i kako se na sjedištu ponašaju. Vaš je zadatak
opisati arhitekturu i razviti sustav koji će administratorima olakšati
upravljanje korisnicima, koristeći mehanizme koje osiguravaju protokol HTTP i
odgovarajuće funkcionalnosti preglednika.

% TODO: Add keywords
\kljucnerijeci{Ključne riječi, odvojene zarezima.}
\end{sazetak}

\engtitle{Tracking the Activity of Online Spellchecker Users}
\begin{abstract}
Abstract.

\keywords{Keywords.}
\end{abstract}

\end{document}
